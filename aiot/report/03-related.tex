\section{Related Works}

While time tracking for athletes competing in sport events is a well known and well understood problem \cite{real-time-tracker}, we thought about finding a solution to provide time tracking for other common tasks using an IoT device.

During the research and design process, we figured out that there are already some time tracking systems that are based on physical devices on the market. So time tracking based on an IoT device is not a new idea. The different projects, which are quickly introduced hereby, are all based on a similar approach. Their implementations, however, are slightly different:

TimeFlip is a similar project \cite{timeflip} based on a cube-like device. The company offers different options, with six, eight and twelve sided cubes.

Timeular \cite{timeular} is also a cube-like tracking device, but uses eight blank sides, which can be labeled individually using a felt-tip pen.

Tiller \cite{tiller} uses a different hardware design. Instead of a cube, it is based on swiping wheel. There is no visible physical information concerning the current activity, which requires a software to be displayed when changing activities. One of the goals of the TimeCube project is to avoid such context switches.

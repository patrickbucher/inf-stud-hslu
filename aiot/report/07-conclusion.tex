\section{Conclusion}

The prototype is working, and time can be tracked and visualized as intended. Every side of the cube represents a project, which can be defined over a web interface. Although more optimization is certainly possible to allow data collection and reporting over an extended period of time, like, for example by aggregating the data before storing it, or reducing the frequency of the measurements being sent to the cloud database.

Making the frequency  of the data collection configurable would allow for additional use cases, where more accurate time tracking is required. In that manner, the TimeCube could also be used as a stopwatch.

For a commercial use, there should be an authentication mechanism, both per device for time tracking, and for connecting to the  web interface. Data could be filtered/grouped for certain users, which would allow for reporting of multiple people or entire teams.

Currently, each cube is identified by a UUID, and each cube is meant to be used by a different person. Additional changes could be made to allow sharing a cube between different people, or to allow the use of multiple cubes by the same person. To increase the ease of use, the different sides of the cube should be made clearly identifiable as belonging to a specific project. This could be done using labels or stickers, as it has been done for the prototypes, or better by adding displays to the sides in order to show the currently configured activities, which would greatly improve the user experience.

\clearpage

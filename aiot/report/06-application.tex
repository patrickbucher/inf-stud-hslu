\section{Application(s)}

There are two main applications of the TimeCube, both usable from the web interface:

\begin{description}
    \item[Configuration] The user can configure the meaning of the different sides of the cube, or use the predefined values. The configured values are stored in the web application and specific to the individual TimeCube.
    \item[Reporting] Reports in the form of tables and bar graphs can be generated for each connected TimeCube. Using the web interface, the user can choose which time period to view.
\end{description}

Time tracking is not only limited for work-related projects. Tracking the amount of time one is spending on personal projects or other pastimes is a possible foundation for self development. If somebody would like to spend more time reading or exercising, but less time watching TV or surfing the web, tracking those activities would be a good start.

\section{Conclusion}

The SCD30 turned out to be a very precise and useful sensor. ZeroMQ offers a lot of facilities that first need to be tried out and understood in order to implement M-to-N relationship between the sensor stations (Arduino or similar devices) and the backends (InfluxDB or others).

The system implemented provides a simple but solid framework to track various variables, be it in an office environment, or elsewhere. In order to be used productively for multiple workplaces, the data processing on the Raspberry Pi needs to be enhanced. Furthermore, an identity of the data source needs to be stored with the measurements on InfluxDB.

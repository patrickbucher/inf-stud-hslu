\section{Introduction}

Many factors have an influence on an employees well-being at the workplace. Some are subjective, others can be measured precisely with sensors.

The right temperature is important. Is it too cold, or too warm? Temperature is subjectivly felt more intensly as the humidity gets higher. 28 $^{\circ}$C might still feel comfortable in dry conditions, but not so as the air's water saturation approaches 100\%.\footnote{\textit{«Cold air with high relative humidity ‹feels› colder than dry air of the same temperature because high humidity in cold weather increases the conduction of heat from the body. Conversely, hot air attended by high relative humidity ‹feels› warmer than it actually is because of an increased conduction of heat to the body combined with a lessening of the cooling effect afforded by evaporation.»} (Source: \url{https://www.infoplease.com/encyclopedia/earth/weather/concepts/humidity})} It is harder to focus on the work if the air quality is bad. High levels of CO₂ might cause drowsiness, and even cause headaches or nausea.\footnote{\url{https://www.kane.co.uk/knowledge-centre/what-are-safe-levels-of-co-and-co2-in-rooms}} Interruptions can be harmful if one wants to focus on the work at hand. If too many of them occur in a given period, the work done soon approaches zero.

The idea of the IoT \textit{WellBean} project is to track those parameters mentioned, as well as a subjective assessment of the well-being perceived by the employee, in real-time. The collected and visualized data allows to gain insight concerning the relationship between objective measurements (air quality, temperature) and the perceived well-being.

Such questions could be: \textit{What temperatures and humidity levels do employees consider «good»?} \textit{What is the effect on interruptions on the employee's well-being?} \textit{Is it possible that interruptions cause bad air quality due to additional people in the room?}

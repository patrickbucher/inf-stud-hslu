\section{Major Milestones \& Deliverables}

\subsection{Team and Roles}

\begin{tabular}{p{9cm}p{4cm}}
\textbf{Project Work Packages} & \textbf{Owner} \\
\hline
Designing the Protocol (Arduino to Raspberry Pi) & Adrian Althaus \\
Testing 6V CO₂ and VOC sensor & Adrian Althaus \\
Testing various sensors & Benno Kuhn \\
Soldering SCD30 & Patrick Bucher \\
Implementing Data Processing on the Raspberry Pi & Patrick Bucher \\
Configuring Grafana & Pascal Kiser \\
Designing the Logo & Pascal Kiser \\
Printing T-Shirts & Patrick Bucher
\end{tabular}

\subsection{Project Planning, Timelines, Milestones \& Deliverables}
Our team did the planning and group organization during the first half of the project quite freely. Every group member possesses at least an Arduino or a Raspberry Pi, some of us also have sensors and actors for IoT devices. The first step consisted of gathering particular topics each member is interested in such as visualization, message queues or inter-device communication or electronics in general, which made it easy to assign the role and work packages to earch member. The group has done individual research and testing on their own devices by project in the first five weeks. Some team members bought additional sensors or displays, just for the sake of testing and playing around. 

Afterwards, the idea of final architecture has become much clearer. Yet having two team members working independently on the goal using a different approach such as CoAP vs. HTTP communication has occurred througout the project. The decision whether or not a contribution will be part of the final system was also based on time resources. In terms of milestones, the team wanted to create a prototype consisting of a CO₂ sensor, Button, Potentiometer, RGB-LED, Arduino and a Raspberry Pi persisting all incoming sensor data in a local InfluxDB by November 29\textsuperscript{th}. 

The team used Gitlab\footnote{\url{https://gitlab.enterpriselab.ch/IoT-I_BA_IOT/i_ba_iot_h19/group02}} for all kinds of artifacts throughout the project.

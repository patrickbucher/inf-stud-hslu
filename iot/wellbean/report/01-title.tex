
\author{Group 2 (A. Althaus, P. Bucher, P. Kiser, B. Kuhn)}
\title{ \vspace{-2cm}WellBean}
\subtitle{HSLU, Informatik Department, Switzerland}
\date{\today}
\maketitle

\section*{Group Members}

\begin{multicols}{2}
\noindent
Adrian Althaus\\ \href{mailto:adrian.althaus@stud.hslu.ch}{adrian.althaus@stud.hslu.ch}

\noindent
Patrick Bucher\\ \href{mailto:patrick.bucher@stud.hslu.ch}{patrick.bucher@stud.hslu.ch}

\noindent
Pascal Kiser\\ \href{mailto:pascal.kiser@stud.hslu.ch}{pascal.kiser@stud.hslu.ch}

\noindent
Benno Kuhn\\ \href{mailto:benno.kuhn@stud.hslu.ch}{benno.kuhn@stud.hslu.ch}
\end{multicols}
\enlargethispage{1cm}
\renewcommand{\baselinestretch}{1.05}\normalsize
\tableofcontents
\renewcommand{\baselinestretch}{1.25}\normalsize

\section{Abstract}
The \textit{WellBean} project aims to monitor and assess objective and
subjective parameters that have an influence on the quality of a workplace.
Objective parameters, such as air quality (CO₂ concentration), temperature, and
humidity are measured and monitored automatically. Subjective parameters, such
as the perceived well-being, and interruptions from co-workers, can be tracked
manually. Correlations and visualizations based on these parameters make it
possible to gain insight on what measurements are considered to be beneficial
to the workplace quality, and, thus, to overall workplace performance.

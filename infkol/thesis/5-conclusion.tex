\section{Conclusion}

OAuth 2 is currently the de facto standard to secure web applications. Even though the OAuth 2 core specification does not provide a bullet-proof recipe for building safe web applications, it is a good starting point, for there is a huge ecosystem consisting of supplementary protocols, techniques, libraries, frameworks, books, articles, best practices and ready to use components. One of the strengths of OAuth 2 is actually that is does \textit{not} prescribe or define implementation details such as cryptographic algorithms, which might be subject to change as time goes. Like an organism that lives longer than its cells, OAuth 2 might be still around after its current building blocks have been deprecated.

The case study shows that a simple OAuth 2 deployment can be implemented with off-the-shelf programming tools. A good HTTP library (for both client and server) combined with some basic crypto functionality is all it needs, even though it is highly recommended to use higher-level abstractions or ready-made components when implementing productive applications. (Some techniques applied in the case study are clear security red flags, such as storing passwords in cleartext, or communicating over plain HTTP.) Since the focus of the example code is only on the communication flow, the case study still makes a point: A simple web application can be sufficiently secured with OAuth 2 using simple tools, even though it takes some effort to understand the intricacies of each and every step.

Unlike the case study, real-world applications do not use opaque random strings as tokens, but encode the delegated access rights into a cryptographically secured token format called JWT. This common token format not only makes it easier to implement OAuth 2 applications (thanks to well-tested and convenient libraries), but also helps to decouple the protected resource from the authorization server. (In the case study, the protected resource still relied on the authorization server to validate tokens.)

Since OAuth 2 lives in the context of the modern web, it is subject to common security vulnerabilities such as session hijacking and cross-site scripting. Common vulnerabilities have common, time-proofed and well understood mitigations, and the OAuth 2 specification and its related documents offer good advice on how to harden web applications against those threats.

As the web becomes the most common way of deploying applications, more sensitive data is moved into web applications, and as OAuth 2 stays the state of the art standard for securing web applications, a solid understanding of OAuth 2 is a necessity for professional web developers nowadays.

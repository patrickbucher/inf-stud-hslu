\documentclass[a4paper,11pt]{scrartcl}

\usepackage[ngerman]{babel}
\usepackage{fontspec}
    \setmainfont{EB Garamond}
    \setsansfont{Open Sans}
\usepackage{multicol}

\begin{document}

\title{Kreatives Schreiben: Nachlese}
\subtitle{zu Dominic Oppligers «acht schtumpfo züri empfernt»}
\author{P. Bucher, M. Della Chiesa, R. Käch, M. Rey}
\date{Mittwoch, 2. Mai 2018}
\maketitle 
\renewcommand{\baselinestretch}{1.5}

\section*{Aufgabenstellung}

Lesen Sie sich die vier untenstehenden Textpassagen aus \textit{acht schtumpfo züri empfernt}\footnote{Oppliger, D. (2018). \textit{acht schtumpfo züri empfernt}. Luzern: Der gesunde Menschenversand, ISBN 978-3-03853-069-5} durch, wahlweise laut im Plenum. Setzen Sie anschliessend die Geschichte fort. Schreiben Sie dabei auf Schweizerdeutsch, entweder in Ihrem eigenen Dialekt -- oder den Stil Dominic Oppligers imitierend auf Zürideutsch. Sie können auch eine eigene Geschichte schreiben.

\section*{Textpassagen}

\subsection*{en baanhofsplatz (S. 8-9)}

\begin{multicols}{2}
\noindent
dasinzo schteikuader xi\\
\\
so chnühöchi\\
graui betonblök\\
ufem ganze plaz ferschtroit\\
\\
zum druf size\\
oder filicht ä nur zum drilaufe\\
oder nözschnäll laufe\\
\\
uf so eim bini ghoket\\
und han ide abixune uf traffi gwartet\\
\\
sisch warm xi\\
de schtei äno\\
\\
sind lüt überde plaz gloffe\\
\\
mitrolkofer und ruksek\\
zudetaxi umpüss\\
unzumiigang\\
\\
forem iigang sind lüt umegschtande\\
hänggwartet\\
\\
paar hänggraucht\\
ti einte imzüg umegluegt\\
anderi ufiri händis\\
\\
e skeilein gizjascho sonenart\\
\\
aber e ganzi zigilang\\
luegschtiä ezänöd a
\end{multicols}

\subsection*{de joschi (S. 30-31)}

\begin{multicols}{2}
\noindent
undeini\\
fo oisne letschte begägnige\\
ischtänn xi\\
\\
woni so mittöppe nünzäni\\
im kaff woni ufgwachse bin\\
in bus gschtige bin\\
\\
i sonen nigelnagelnoie\\
usepüzlete bus\\
so eine fode erschte\\
mizo biltschirm\\
wo irgendwelchi nius azeigt werded\\
\\
kürzt\\
umpfölig sinfrey\\
meischtens so mit schriipfäler\\
\\
undichpi döt wisawi\\
fosomene junge tüp\\
im fiärerapteil apghoket\\
\\
untä tüp häzeelig\\
ummipfüürrotenauge\\
for sich anexmeilet\\
\\
hätte schporttäsche debi gha\\
wo u huere nach gras gschtunke hätt\\
\\
untänn schtiigte joschi ii\\
unzizt zu ois ane
\end{multicols}

\subsection*{wuaiörrutine (S. 91-93)}

\begin{multicols}{2}
\noindent
weldas ischebeneso xi\\
dasi am choche xi bin\\
\\
und woni en awoggadochern\\
ha welenin komposchkübel\\
ufem fänschterbank legge\\
\\
hani wi immer no gschnäll\\
rutinemässig\\
durde hof\\
as geissbergersche huus anegluegt\\
\\
untaxeeni\\
wi bi dene\\
usem groosse chuchifänschter use\\
de älteri bueb useluegt\\
\\
jaso richtig usehanget\\
unzu sim roote ball abeluegt\\
wo im hof une glägenisch\\
\\
underhäzich blizschnäll umtreyt\\
undisch defo\\
tschtäge durap\\
in hof\\
zum ball\\
\\
unz chuchifänschter\\
hätter schperangelwiit offeglaa\\
untännebe xeeni\\
de chopf fom chline meitli\\
plözli im fänschter erschiine\\
\\
unxeesi\\
ufs fänschterbrätt ufeschtiige\\
unzekunde schpöter\\
im fänschter schtaa\\
\\
untas meitli isch öppeneinehalb xi\\
oder höchschtens zwei\\
\\
unzhäzich füreglänt\\
unzum ball abe\\
in hof glueget
\end{multicols}

\subsection*{am frauetag (S. 130-132)}
\begin{multicols}{2}
\noindent
undichpino rächt lang ade bar plibe\\
und ha nüssli gässe\\
und eis biär\\
nachem andere trunke\\
\\
und woni dänn mal ufs weezee bi\\
hani bim laufe scho gmerkt\\
wi psoffe dasi bin\\
\\
umpi dussenim tunkle\\
zude weezees übere\\
\\
unteetisch\\
schone huere langi schlange xi\\
\\
aber wel nur fraue ide schlange gschtande sind\\
hani gmeint\\
essegi tschlange fom fraueweezee\\
\\
undichpi eifach\\
ade schlange ferbi gloffe\\
\\
und woni fore xi bin\\
isch grad eini\\
fode beide weezeetüre ufgange\\
\\
untisch en maa usecho\\
\\
undichpininegschlüpft\\
und hanapschlosse\\
und grad losgleit\\
\\
und ha scho ghört\\
dass dusse\\
irgenzochli en tumult losgangenisch\\
\\
hanaber tänkt\\
s heg nüd mipmiär ztue\\
\\
und woni wider\\
usem weezee usecho bi\\
\\
luegepmi öppe zä\\
huere bösi xichter a
\end{multicols}

\end{document}

\section*{Management Summary}

Dieses Dokument wurde im Rahmen des Moduls PREN 2 im Frühlingssemester 2018 von der Gruppe 7 erstellt und dokumentiert die Umsetzung der autonomen Laufkatze \textit{Silisloth}, wie sie im Herbstsemester 2017 im Rahmen des Moduls PREN 1 konzipiert worden ist. \textit{Silisloth} ist ein Gerät, das an einem Seil hängend nach erteiltem Startsignal eine Last lokalisieren, aufnehmen, über Hindernisse hinweg befördern und in einem durch konzentrische Quadrate markierten Zielbereich absetzen kann. Die hier vorliegende Dokumentation beschreibt die Umsetzung der Bereiche Mechanik (Aufbau, Montage), Elektronik (Schaltungen, Verdrahtung, Motorensteuerung), Informatik (Bildverarbeitung, Steuerung, Smartphone-App) wie auch interdisziplinäre Fragestellungen (Kommunikation zwischen den Geräten, vorgenommene Tests) sowie den geplanten Ablauf beim Absolvieren der gestellten Wettbewerbsaufgabe. Wie schon das zugrundeliegende Konzept \cite{pren1}, auf das in diesem Dokument bei Bedarf verwiesen wird, folgt die Umsetzung der Maxime «eine sichere, zuverlässige und für alle verständliche Lösung zu erarbeiten, die dennoch originell und für alle Beteiligten lehrreich ist und zur Zusammenarbeit über die Fachgrenzen hinweg anregt», wie sie in PREN 1 als Gruppenziel formuliert worden ist.

\section*{Management Summary}

Dieses Dokument wurde ihm Rahmen des Moduls PREN 1 im Herbstsemester 2017 von der Gruppe 7 erstellt und beschreibt ein Konzept für eine autonome Laufkatze namens \textit{Silisloth} , wie sie im Frühlingssemester 2018 im Rahmen des Moduls PREN 2 umgesetzt werden soll. \textit{Silisloth} ist ein Gerät, das an einem Seil hängend nach erteiltem Startsignal eine Last lokalisieren, aufnehmen, über Hindernisse hinweg befördern und in einem durch konzentrische Rechtecke markierten Zielbereich absetzen soll. Das hier beschriebene Konzept erörtert Fragestellungen der Mechanik (Aufhängung, Greifmechanismus, Gehäuse), der Elektrotechnik (Sensorik, Stromversorgung, Mikrocontroller) und der Informatik (Bildverarbeitung, Entwicklerboard, Softwarearchitektur). Die einzelnen Komponenten sind darin genauer beschrieben; Tests mit Prototypen dieser Komponenten dokumentiert. Weiter beschreibt das Konzept die Kombination der einzelnen Komponenten zu einem autonomen Gerät. Das Konzept folgt der Maxime «eine sichere, zuverlässige und für alle verständliche Lösung [zu] erarbeiten» (siehe \appref{app:gruppenziele}), die dennoch originell und für alle Beteiligten lehrreich ist und zur Zusammenarbeit über die Fachgrenzen hinweg anregt.

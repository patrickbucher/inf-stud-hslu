\section{Einleitung}

Das vorliegende Dokument mit Anhängen wurde von der Gruppe 7 im Rahmen des Projektmoduls «Produktentwicklung» an der Hochschule Luzern ‒ Technik \& Architektur erstellt. Es stellt einerseits das Projekt-Zwischenergebnis nach dem Herbstsemester 2017 und dem ersten Modulteil PREN 1 dar, soll andererseits im Frühlingssemester 2018 im zweiten Modulteil PREN 2 als Grundlage für die Umsetzung des im Folgenden beschriebenen Geräts dienen: einer autonomen Laufkatze namens \textit{Silisloth}. Der Name setzt sich aus «Silikon» und «Sloth» zusammen, denn \textit{Silisloth} arbeitet wie ein Faultier kopfüber hängend und verfügt über eine wichtige Komponente aus Silikon.

Grundlage für das Projekt ist der im \appref{app:projektauftrag} angeführte Projektauftrag. Dieser beschreibt die Rahmenbedingungen der zu erstellenden Apparatur und des Wettbewerbs, an dem diese im Sommer 2018 antreten soll. \textit{Silisloth} soll an einem Seil hängend eine Last in der Form eines Holzwürfels erkennen, aufnehmen, anheben, dem ansteigenden Seil entlang fahrend über zufällig positionierte Hindernisse hinwegführen und in einem markiertem Zielbereich möglichst genau absetzen ‒ und schliesslich zum Endmast fahren und anhalten. Der Vorgang muss in einem gesetzten Zeitrahmen abgeschlossen sein; auch die Vorbereitungszeit und Entwicklungskosten sind begrenzt.

Im ersten Teil wird der Projektauftrag analysiert (\secref{sec:analyse-aufgabenstellung}) und daraus eine Anforderungsliste (\appref{app:anforderungsliste}) abgeleitet. Das Gesamtkonzept (\secref{sec:gesamtkonzept}) bietet einen Überblick über die Bauweise von \textit{Silisloth}. In den weiteren Abschnitten (\secrefplain{sec:komponenten-maschinenbau}; \secrefplain{sec:komponenten-elektrotechnik} und \secrefplain{sec:komponenten-informatik}) werden die einzelnen Komponenten erklärt. Diese wurden mithilfe einer Recherche (\appref{app:recherche}) ermittelt, teils in Versuchen (\appref{app:versuche}) erprobt und kombiniert zu verschiedenen Konzeptvarianten (\appref{app:konzeptvarianten}) evaluiert. Da die Wahl einiger Komponenten mit einem hohen Risiko verbunden ist (\secref{sec:risikomanagement}), wurden Alternativkomponenten ermittelt (\secref{sec:komponenten-alternativen}).

Das Projektmanagement umfasst neben dem Projektplan (\appref{app:projektplan}) auch die Bereiche Risikomanagement (\secref{sec:risikomanagement}) und Budgetplanung (\secref{sec:budget}). Im Schlusswort (\secref{sec:schlusswort}) wird neben dem weiteren Vorgehen und einem kurzen Fazit auf die \textit{«Lessons learned»} eingegangen; \textit{Silisloth} ist zwar das eigentliche Projektziel, der Lerneffekt aber der übergeordnete Zweck dieser Projektarbeit.

Die Gruppe 7 hat sich zu Beginn des Moduls PREN 1 Ziele gesetzt (siehe \appref{app:gruppenziele}), die bei der Entwicklung des hier beschriebenen Lösungsvorschlags mitberücksichtigt wurden. Dabei ist ein Vorschlag für eine \textit{«sichere, zuverlässige und für alle verständliche Lösung»} (Ziel 5) entstanden, wobei durch die Wahl innovativer Komponenten der \textit{«Spass an der Arbeit»} (Ziel 2) nicht zu kurz kam, fachbereichübergreifend etwas Neues gelernt wurde (Ziel 3) und aufgrund einer sinnvollen Aufgabenverteilung sichergestellt wurde, dass \textit{«jedes Gruppenmitglied seinen Beitrag leistet»}. Dies soll auch für das Modul PREN 2 gelten, das die Gruppe 7 in gleicher Zusammensetzung bestreiten will.

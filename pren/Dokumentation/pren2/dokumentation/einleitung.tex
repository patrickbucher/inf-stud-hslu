\section{Einleitung}

In PREN 1 wurde das Konzept für \textit{Silisloth} -- eine autonome Laufkatze -- entwickelt. In PREN 2 wurde dieses umgesetzt. Die verwendeten und entwickelten Komponenten und die Konstruktion folgen dabei grösstenteils diesem Konzept. Das vorliegende Dokument gibt einen Überblick über den dabei erstellten Prototypen und handelt von dessen Umsetzung. Änderungen gegenüber dem ursprünglichen Konzept werden dabei gesondert hervorgehoben. Um Wiederholungen und das Kopieren von Passagen zu vermeiden, wird bei Bedarf auf das Konzept \cite{pren1} verwiesen.

Die Konstruktion bildet die Grundlage von \textit{Silisloth}. Die schlanke Lösung, die den Prinzipien der Leichtbauweise folgt, wird in Abschnitt \secrefplain{sec:konstruktion} beschrieben. Eine Explosionsansicht des CAD-Modells bietet einen Überblick über die einzelnen Komponenten und deren Anordnung.

In Abschnitt \secrefplain{sec:elektronik} geht es einerseits um verschiedene Elektrotechnikkomponenten, die im Konzept von PREN 1 noch nicht erwähnt wurden, andererseits um die Verdrahtung der elektronischen Komponenten miteinander. Last- und Steuerstromkreis werden in einem Blockschema dargestellt.

In Abschnitt \secrefplain{sec:ablauf} wird das Absolvieren der Wettbewerbsaufgabe aus einer technischen Perspektive erläutert. Der Leser erhält dadurch einen Einblick in die Funktionsweise von \textit{Silisloth}, und erfährt, welche Überlegungen gemacht wurden, um das gestellte Problem erfolgreich zu lösen. Für Implementierungsdetails wird auf später folgende Abschnitte verwiesen.

\textit{Silisloth} ist mehr als ein elektromechanisches Gerät. Es ist mit einem Raspberry Pi (fortan «Raspi» genannt) und einem Arduino ausgestattet, was schon genügt um es als fahrendes Rechennetzwerk zu bezeichnen. Das Startsignal wird von einer Smartphone-App aus erteilt, welche auch die Lastkoordinaten anzeigt. Ein externer Laptop dient dazu den Ablauf des Raspi-Programms zu überwachen -- und um \textit{Silisloth} in Bereitschaft zu versetzen. Die verschiedenen Kommunikationsprotokolle, die im Zusammenspiel dieser vier Computer zum Einsatz kommen, werden in Abschnitt \secrefplain{sec:kommunikation} beschrieben. Zur Veranschaulichung wird ein UML-Sequenzdiagramm verwendet.

Neben der Kommunikation stellt das Erkennen des Zielfeldes und die Ermittlung des Abstandes zu diesem eine grosse Herausforderungen dar. Der Abschnitt \secrefplain{sec:bildverarbeitung} geht ausführlich auf die erarbeitete Lösung ein. Dabei wird zwar auch die dabei verwendete OpenCV-Library und die Programmiersprache Python angesprochen, die Erläuterungen sollen aber dennoch für einen Leser verständlich sein, der in diesen Technologien nicht bewandert ist. Eine Bildserie des Zielfeldes vermittelt dem Leser die Perspektive der Kamera, was die erstellte Lösung bessser nachvollziehbar macht.

\textit{Silisloth} erfordert zum Betrieb verschiedene Softwarekomponenten. Die Smartphone-App zur Erteilung des Startsignals wird in Abschnitt \secrefplain{sec:smartphoneapp} erläutert. Die ganze Programmlogik, die \textit{Silisloth} zu einem autonomen Gerät machen, ist auf einen Raspi und einen Arduino verteilt. Abschnitt \secrefplain{sec:raspi} geht auf die Python-Lösung ein, welche einerseits die Aktionen von \textit{Silisloth} koordiniert und andererseits verschiedene Hardware- und Kommunikationsschnittstellen bedient. Ein UML-Klassendiagramm verschafft Übersicht über das erstellte Python-Projekt. Der Abschnitt \secrefplain{sec:arduino} handelt von der in C und mit der RTOS-Library geschriebenen Softwarelösung, welche in erster Linie für die Motorensteuerung zuständig ist, jedoch auch Kommunikations- und Koordinationsprobleme löst. Besonders wichtig sind hier interne Stati, welche mit einer State-Machine veranschaulicht werden.

Der in PREN 2 erarbeitete Prototyp soll, wie schon das in PREN 1 erstelle Konzept, den gestellten Anforderungen genügen. Inwiefern diese Anforderungen erfüllt sind, und wie der Prototyp mittels Tests kontinuierlich verbessert wurde, ist Thema von Abschnitt \secrefplain{sec:zielerreichung}.

Im Abschnitt \secrefplain{sec:schlusswort} wird die Umsetzung des \textit{Silisloth}-Projektes reflektiert. Dabei wird einerseits auf die einzelnen Projektphasen in der Umsetzung, andererseits auf bestimmte dabei gemachte Erfahrungen eingegangen. Am Schluss wird ein Fazit gezogen, welches -- soviel vorweg -- aufgrund der gelungenen Lösung positiv ausfällt.

\subsection{Verwendete Werkzeuge}

Die Arbeit wurde mit \LaTeX\footnote{\url{https://www.latex-project.org/}} und \XeLaTeX\footnote{\url{https://www.sharelatex.com/learn/XeLaTeX}} in den Schriftarten EB Garamond, \textsf{Open Sans} und \texttt{Fira Mono} gesetzt. Die Grafiken wurden mit \textit{Graphviz}\footnote{\url{https://www.graphviz.org/}}, \textit{plantuml}\footnote{\url{http://plantuml.com/}} und \textit{LibreOffice Draw}\footnote{\url{https://www.libreoffice.org/discover/draw/}} erstellt. Weitere Hilfsmittel waren \texttt{git}\footnote{\url{https://git-scm.com/}} zur Versionsverwaltung der Quelldateien und \texttt{make}\footnote{\url{https://www.gnu.org/software/make/}} zum Erstellen der Grafiken und des Dokuments aus den Quelldateien.

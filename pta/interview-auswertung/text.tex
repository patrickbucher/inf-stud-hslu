\documentclass[11pt,a4paper]{scrartcl}
\usepackage{apacite}
\usepackage{ngerman}
\usepackage[T1]{fontenc}
\usepackage[utf8]{inputenc}
\usepackage{times}
\usepackage{url}
\setcounter{secnumdepth}{0}

\begin{document}

\title{Mitfahrgelegenheit}
\subtitle{Interviewauswertung}
\author{Patrick Bucher}
<<<<<<< HEAD
\date{12. November 2016}
=======
\date{9. November 2016}
>>>>>>> 95566dbcc98877a11cd62ec304a6860bb6e37909
\maketitle

\section{Mitfahrgelegenheit}
Um mehr zum Thema Mitfahrgelegenheit zu erfahren, hat die Gruppe 7 ein Interview mit einer Person durchgeführt, die schon viele Erfahrungen in diesem Bereich gemacht hat. Wir haben uns am 2. November 2016 mit Richard B., 57, im Hotel Radisson in Luzern getroffen und ein Interview mit ihm geführt.

Seine ersten Erfahrungen mit Mitfahrgelegenheiten habe Herr B. vor fünf Jahren gemacht, als er wöchentlich von München nach Luxemburg und zurück gependelt sei. Damals habe er selber mitfahren wollen, habe aber keine entsprechende Mitfahrgelegenheit gefunden. Darum sei er dann weiterhin selber gefahren und habe sich selber als Mitfahrgelegenheit angeboten. \cite[Zeilen 2-6]{interview} Mitreisende habe er über die Plattform Mitfahrgelegenheit(en).de, die seit einem Jahr BlaBlaCar heisse\footnote{Das französische Start-Up BlaBlaCar hatte die deutsche Firma carpooling.com übernommen, die unter anderem das Portal Mitfahrgelegenheit.de betrieb. \cite{afp}. }, gefunden. \cite[Zeilen 11-12]{interview}

\subsection{Anreize}

Herr B. erwähnte zwei Gründe, warum er sich als Mitfahrgelegenheit angeboten habe. Das sei zum einen der finanzielle Aspekt gewesen, denn durch die entrichtete Mitreisegebühr habe er jeweils das verbrauchte Benzin bezahlen können. \cite[Zeilen 20; 53]{interview} Zum anderen sei es ihm lieber gewesen, auf einer längeren Fahrt einen Mitfahrer zur Unterhaltung zu haben, wobei er auch interessante Leute kennengelernt habe. \cite[Zeilen 21-22; 30-32]{interview} Auf Nachfrage hin bezeichnet er auch den ökologischen Aspekt als wichtig \cite[Zeilen 24-27]{interview}, ein Aspekt, der auch von BlaBlaCar herausgestrichen wird: so würden Mitfahrer durch eine Partnerschaft mit der gemeinnützigen Organisation myclimate zu 100\% $CO_2$-neutral reisen. \cite{blablacar}

\subsection{Bezahlung}

Zu Beginn sei die Bezahlung durch die Mitfahrenden direkt erfolgt. \cite[Zeile 12]{interview}. Als Mitfahrgelegenheit.de am 27. März 2013 eine Gebühr für Strecken ab 100 Kilometern «aufgrund der Personalkosten für die 60 Mitarbeiter» eingeführt hatte, sollen die Benutzer damit angefangen haben, das Reservierungssystem zu umgehen. Die Telefonnummern der Mitfahrenden und Fahrer sollen erst nach erfolgter Buchung verfügbar gewesen sein. Die Mitfahrer sollen Reisen gebucht und sogleich storniert haben, waren dann aber im Besitz der Kontaktdaten, wodurch die Abrechnung weiterhin direkt und unter Ausschluss der Plattform erfolgt sein soll. Die Benutzer seien auf der Suche nach einer kostenlosen Alternative unter anderem beim französischen Anbieter BlaBlaCar fündig geworden, der das Portal Mitfahrgelegenheit.de später auch übernommen hat. Über die Einführung eines eigenen Bezahlsystems habe aber auch BlaBlaCar Gebühren eingeführt \cite{zeit}, die mit 10 Prozent zu Buche schlügen. \cite[Zeilen 12-13]{interview}

Auf internationalen Strecken habe Herr B. aber immer den vollen Betrag erhalten, da das Portal bei Auslandsfahrten nicht partizipiert habe. \cite[Zeilen 72-73]{interview} Als Kostenvorschlag gab das Portal die Faustregel 6 Euro pro 100 Kilometer vor. \cite[Zeilen 67-68]{interview} Für die 400 Kilometer von Luxemburg nach München (und später von Luzern nach München) habe Herr B. darum 24 Euro verlangt. \cite[Zeilen 67-68]{interview} Mitgenommen habe er meistens drei, manchmal auch vier Leute, insgesamt schätzt er die Zahl der Mitreisenden für beide Strecken (München-Luzern, München-Luxemburg) auf jeweils 200 bis 300, wobei die meisten Studenten gewesen sein sollen.\cite[Zeilen 55-56]{interview} Insgesamt hat Herr B. also 9’600-14’400 Euro für Fahrten, die er sowieso hätte bestreiten müssen, eingenommen. Die Frage, ob und wie diese Einnahmen zu versteuern seien, wurde am Interview nicht erörtert.

\subsection{Weitere Erfahrungen}

Unter sozialen («jemanden zum Reden dabei haben») und finanziellen («das Spritgeld reinkriegen») Gesichtspunkten hat es sich für Herrn B. gelohnt, sich als Mitfahrgelegenheit anzubieten \cite[Zeilen 49-53]{interview}. Ungefähr 15 Prozent der Mitreisenden sollen eine Bewertung abgegeben haben, wobei er selber ungefähr 45 Bewertungen bekommen habe. Auf der Skala von einem bis fünf Sterne soll er bis auf zwei Ausnahmen fünf Sterne erhalten haben. Ein Mitreisender soll ihm nur drei Sterne gegeben haben, da die Mitreisenden unverhofft hinten zu dritt hätten sitzen müssen. \cite[Zeilen 42-46]{interview} Weiter sei es mit einer Mitreisenden bzw. deren Freundin nach einem Missverständnis und daraus folgenden Verspätung zu einer verbalen Auseinandersetzung gekommen, wonach Herr B. die Frau nicht mitgenommen habe, worauf er mit nur einem Stern bewertet worden sei. \cite[Zeilen 86-104]{interview}

Selbst die wenigen schlechten Bewertungen hätten für ihn negative Folgen gehabt. So wurde Herr B. aufgrund seiner «schlechten» Bewertung selber einmal nicht mitgenommen, als er Gebrauch von einer Mitfahrgelegenheit machen wollte. \cite[Zeilen 152-159]{interview} Grundsätzlich müsse man sehr kundenorientiert sein, meinte Herr B. dazu. \cite[Zeilen 162-168]{interview}

Bei einer Fahrt von Luxemburg nach München sei es zu einem Missverständnis mit zwei indischen Mitreisenden gekommen. Diese hätten nicht in Luxemburg, sondern beim vereinbarten Zwischenhalt in Karlsruhe gewartet ‒ jedoch zwei Stunden zu früh, wonach die beiden Inder mit dem Zug und Herr B. alleine im Auto gereist seien. \cite[Zeilen 76-85]{interview}

Einmal habe er einen älteren, durchaus reichen Herrn von München nach Trier mitgenommen, wo er seinen neuen Porsche Cayenne abholen wollte. Der finanzielle Aspekt sei diesem Mitreisenden wohl nicht so wichtig gewesen, vielmehr wollte er nicht mit der Bahn reisen, und sein Sohn solle ihm die Mitfahrgelegenheit empfohlen haben. \cite[Zeilen 32-37]{interview}

Herr B. fühlte sich manchmal wie ein Taxidienst behandelt. Grundsätzlich würden Routen von Bahnhof bis Bahnhof vereinbart. Manche Mitreisende sollen aber die Erwartung gehabt haben, von Tür zu Tür transportiert zu werden, was Herr B. nur in manchen Fällen machte. \cite[Zeilen 111-114]{interview}

Sicherheitstechnische Bedenken hatte Herr B. nur in einem Fall, als ein auf ihn etwas suspekt wirkender Mitreisender und dessen Cousin hätten mitreisen wollen. \cite[Zeilen 117-123]{interview} Die Plattform biete einen grundlegenden Versicherungsschutz \cite[Zeilen 126-129]{interview}. Herr B. soll aber noch nie einen Unfall gehabt haben, darum denke er gar nicht so stark über dieses Thema nach. \cite[Zeilen 132-133]{interview}

\subsection{Ansichten zu verwandten Mobilitäts-Themen}

Herr B. wäre grundsätzlich dazu bereit, eine Fahrgemeinschaft auf täglicher Basis mit Mitarbeitenden zu gründen, wobei er es für kurze Strecken für weniger sinnvoll halte, zumal das einem doch recht einschränken könne. \cite[Zeilen 191-195]{interview}

Als Bezahlung der Mitfahrer würde Herr B. grundsätzlich auch Reisegutscheine (z.B. Reka-Checks) akzeptieren, unter der Bedingung, dass diese lange gültig seien. \cite[Zeilen 196-198]{interview}

Mit der Firma Uber habe Herr B. auch schon Erfahrungen gemacht. Diese wollte ihn als Fahrer gewinnen, doch ihm sei der ganze Bewerbungsprozess zu mühsam gewesen. \cite[Zeilen 199-204]{interview}
Sein eigenes Auto würde er lieber nicht an Unbekannte vermieten, denn die Leute gingen zu rücksichtslos mit Autos um, die nicht ihre eigenen sind. Wenn er ein älteres Auto hätte, käme eine Weitervermietung in Frage. \cite[Zeilen 186-190]{interview}

\renewcommand{\refname}{Quellen}
\bibliographystyle{apacite}
\bibliography{bibliographie}

\end{document}
